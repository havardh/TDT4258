\subsection{Sound}

\subsubsection{Test program}
During the development of the {\bf Audio} module the program in /sounds/test was created.

The final test program tests two sounds concurrenlty by playing one first, waiting
for two second and playing one more.

\begin{lstlisting}
int main( void ) {

  char *sample1 = "./data/sample1.wav";
  char *sample2 = "./data/sample2.wav";

  AudioPlay( sample1 );
  sleep( 2 );
  AudioPlay( sample2 );

  return 0;
}

\end{lstlisting}

\subsubsection{Running the test}
The test is compiled with the makefile in /sounds/Makefile. This generates a executable
bin/sounds\_test. This can be executed on the board. Like the graphics driver it has to
have a {\bf data} folder in the same folder as the test is executed from and this must
contain the files {\bf sample1.wav} and {\bf sample2.wav}. These files must be of the
{\bf wave} file format with specifics 8 bit per sample and 8000 samples per minute.

For the sound to be audible the sound setting on the Linux on the board must be set according
to the specifics in section \ref{sec:alsa}.

\subsubsection{Expected results}
The first sample should be audible.

\subsubsection{Test results}
As expected
