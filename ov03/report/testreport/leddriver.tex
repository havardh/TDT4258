\subsection{Led Driver}
%\label{sec:result}
To test the led driver we wrote a simple c program that did three things;
\begin{enumerate}
  \item Initialize the leds, i.e. open the file \textit{/dev/leds} with write permission
  \item Write different values to the led
  \item Close the file
\end{enumerate}
The simple test is testing one light at the time, before it pauses for one secound and
lights the next one.\\
\\
{\bf Test Case 1}: Open leds through a C program \\
{\bf Description}: It should be possible to open and close the device file through a
C program \\
{\bf Precondition}: Board is set up with operating system and the drivers are loaded\\
{\bf Execution}: Run the program, calling the {\it open(file, flag)} function on the device
file., followed by a {\it close} on the file.\\
{\bf Expected Result}: {\it open} should NOT return a negative number, {\it close}
should return 0.\\
{\bf Observed Result}: {\it open} returned a positive number, {\it close}
returned 0. \\
 \\
{\bf Test Case 2}: Write to leds through a C program. LED 0 should light \\
{\bf Description}: The first lED should be on after writing to the device file \\
{\bf Precondition}: Board is set up with operating system and the drivers are loaded,
and device file is opened with the {\it O\_WRONLY} flag.\\
{\bf Execution}: Write the {\it short int} value 1 to the device file \\
{\bf Expected Result}: LED 0 should be ON, all other leds should be OFF. \\
{\bf Observed Result}: LED 0 was the only active LED \\
 \\
{\bf Test Case 3}: Write to leds through the linux terminal\\
{\bf Description}: Write a single byte to the device file using a terminal commando\\
{\bf Precondition}: Board is set up with operating system and the drivers are loaded\\
{\bf Execution}: Run the command {\bf echo A $>$ /dev/leds} \\
{\bf Expected Result}: LED 6 and LED 0 should be ON, all other leds should be OFF.\\
{\bf Observed Result}: LED 6 and LED 0 turned ON, the others turned OFF. We also
observed that the command did not end the execution in the terminal.\\
 \\
There was one problem with the third test case. The exectued command did not seam to end.
While looking over the kernel logs we found that the value {\it 10}, the ascii value for
{\it new line}, was constantly being tried written to the device file. Thus it seams like
the driver is not working fully as expected, but it serves its purpose when writing to it
through a C program.
