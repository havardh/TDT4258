\subsection{Game}

In order to work, the game have to be compiled from source, copied onto the board 
and executed from the right folder.
This can be done from any of the computers in the lab which contains the {\bf avr32-linux-gcc} compiler.

\subsubsection{Compilation}
From the {\it game} directory in the project on the lab computer, the following command will compile the game:
\begin{verbatim}
  make
\end{verbatim}
This produces the file ./game/bin/game which is an executable that can run under 
the linux distribution installed on the SD-card.

\subsubsection{Copying the executable to the board}
From the {\it game} directory on the lab pc the following command copies the game executable to the board.
\begin{verbatim}
 scp bin/game avr32@<IP address>:~/
\end{verbatim}
Where {\it $<$IP address$>$} is the IP obtained in section \ref{sec:boardsetup}.

\subsubsection{Copying the dependecies}
From the /game/data directory all the .bmp and .wav files must be copied to the board. They must
lie in the data folder in the directory of the game executable
\begin{verbatim}
  scp data/* avr32@<IP address>:~/data
\end{verbatim}

\subsubsection{Running the game}
From the /home/avr32 directory on the board the game can now be started with:
\begin{verbatim}
  ./game
\end{verbatim}
In order to play the game the user has to be root, this is obtained by 
running the {\bf su} command
and using password = roota.
