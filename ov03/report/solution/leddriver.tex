\subsection{Led Driver}
The main difference between the two drivers is the use of the file operation
functions, specifically the \textit{write} for leds and the \textit{read} for
buttons.\\
\\
The file\_operation struct in leds registers the four file operations open,
read, write and release, but only the write-function is implemented.
The remaining three functions only return the value 0.\\
\\
The write function is passed several parameters in which the string,
\textit{char *buff} is the only one of interest for this driver.
When the system calls the write\_leds function, the device driver will 
write the first byte of the buffer to the PIOB sodr register, i.e. the register
that holds the led status. Afterwards it will return the value 1, indicating that
one byte was written successfully.
\\
\begin{lstlisting}
static ssize_t write_leds ( struct file *filp, char __user *buff, 
		size_t count, loff_t *offp ) {
	volatile avr32_pio_t *piob = &AVR32_PIOB;
	if ( buff[0] != 10 ) {
		piob->codr = 0xff;
		piob->sodr = buff[0];// & piob->codr;
		printk ( KERN_INFO "Leds written successfully!\n" );
		return 1; // Successfully written 1 byte
	} else {
		return 0;
	}
}
\end{lstlisting}