\subsection{Sound}

\subsubsection{Playing a sound}
The 8 bit wave fileformat is mosty an array of bytes which contains samples that can be written
directly to the ABDAC. It does contain a fileheader shown in the file {\bf wave.h}, but as this is very
small it is not heard when pipeing it to the ABDAC. This fact enabled us to implement a minimal audio
module which just opens a file (the wave) and writes its content directly to the {\bf /dev/dsp}.

\begin{lstlisting}
  static void *PlaySound( void *thread_arg ) {

    char *sample = (char*)thread_arg;

    FILE *dsp_fd = fopen( "/dev/dsp", "wb" );
    FILE *fd = fopen( sample, "rb" );

    if (!dsp_fd) {
      return;
    }

    int c;
    while ( (c = fgetc(fd)) != EOF ) {
      fputc( c, dsp_fd );
    }

    fclose( fd );
    fclose( dsp_fd );
  }

\end{lstlisting}

Here to files are opened, one for the device driver and one for the sound samples. The samplefile
is read and written byte by byte to the ABDAC.

\subsubsection{Threads}
To not conflict with the rest of the program and to keep the design simple each new sample is
played in a new thread. If a new sample request comes while one is already playing it just ignored.
This is done by the {\it if (!dsp\_fd)} in the previous listing. If {\bf fopen} cannot get a hold
of the {\bf /dev/dsp} file it will return zero and the thread will stop without playing the sound.
If {\bf fopen} return a non-zero number the sample will be played before the thread terminates.

The tread is created and passed the samples filename through the {\bf pthread} library call
{\bf pthread\_create}.
\begin{lstlisting}
  void AudioPlay( char *sample ) {

    pthread_t thread;

    int rc = pthread_create(&thread, NULL, PlaySound, (void*) sample );
  }
\end{lstlisting}
