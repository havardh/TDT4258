\subsection{Drivers}
The led driver and the button driver has a lot of similar code, mainly 
the initialization process and the termination process of the driver. 
They are both character devices that registers a memory region on the 
PIOB and registers itself as a character device, with major numbers and 
file operations like read and write. The following sections 
describe the implementation of the led driver, but the procedure is equal for the buttons driver.

\subsubsection{Module initialization}
The leds and buttons driver have the init functions \textbf{leds\_init}
and \textbf{buttons\_init}, respectively. They are registered, and thus 
picked up by the kernel, by calling the function \textbf{module\_init}
passing the init-function as a parameter.
\\
\begin{lstlisting}
module_init ( leds_init );
\end{lstlisting}
Both the drivers have a file\_operations structure,
defined in the C file. All it does is map different file operations,
implemented in the driver, to the ones defined in the structure.
This file\_operations structure will later on be registered together
with the actual device.
\\
\begin{lstlisting}
static struct file_operations led_fops = {
	.owner = THIS_MODULE,
	.open = open_leds,
	.write = write_leds,
	.read = read_leds,
	.release = release_leds
};
\end{lstlisting}
The first thing that happens inside the leds\_init function
is allocation of a character device region
and assigning a major number to the device.
\\
\begin{lstlisting}
static int leds_init ( void ) {
	//...
	result = alloc_chrdev_region ( &dev, leds_minor, 
		NUM_DEVICES, "leds" );
	leds_major = MAJOR ( dev );
	//...
\end{lstlisting}
If alloc\_chrdev\_region returns a negative number there is a problem getting
a major number to the device and the init method returns with an error.
If result is positive we build a dev\_t data item, 
representing the device based on the minor and major number, and requests memory region that we are going to use on the GPIO at the PIOB address.
If all goes well we initialize the \textit{per} and, 
\textit{oer} registers at piob (\textit{per} and \textit{puer}
for buttons), in order to be able to read and write from PIOB.
\\
\begin{lstlisting}
dev = MKDEV ( leds_major, leds_minor );
result = (int) request_region ( AVR32_PIOB_ADDRESS, mem_quantum, "leds" );
// Initialize the leds
volatile avr32_pio_t *piob = &AVR32_PIOB;
piob->per |= 0xff;
piob->oer |= 0xff;
\end{lstlisting}
Finally, the leds\_init function allocates and sets up the character 
device structure \textit{char\_dev}. 
It adds all the led file operations defined in
the struct \textit{led\_fops} to \textit{char\_dev}, and 
it registers the device with the dev\_t \textit{dev} data item.
\\
\begin{lstlisting}
// Set up char_dev structure for the device
struct cdev *char_dev = cdev_alloc ();
cdev_init ( char_dev, &led_fops );
char_dev->owner = THIS_MODULE;
char_dev->ops = &led_fops;
int error = cdev_add (char_dev, dev, 1);
\end{lstlisting}
If there is an error, i.e. \textit{error} is a positive number, it will
be logged to the kernel. Otherwise, the device driver initialization 
should have completed successfully.

\subsubsection{Module termination}
The drivers also have an exit-function, \textbf{leds\_exit} and 
\textbf{buttons\_exit} that is registered the same way as 
the init function. This function is called when the kernel module is 
removed from the operating system. 
\\
\begin{lstlisting}
module_exit ( leds_exit );
\end{lstlisting}
The termination procedure of the kernel module is much smaller than the 
initialization procedure. It basically clears out the output register
on the device, releases the previously requested memory region and
unregisters the previously registered character device region for
the device.
\\
\begin{lstlisting}
static void leds_exit ( void ) {
	int mem_quantum = sizeof( avr32_pio_t );
	volatile avr32_pio_t *piob = &AVR32_PIOB;
	piob->codr = 0xff;
	release_region ( AVR32_PIOB_ADDRESS, mem_quantum );
	unregister_chrdev_region ( dev, NUM_DEVICES );
}
\end{lstlisting}

