\subsection{Graphics}

\subsubsection{Screen}

The screen is opened and memory mapped in the {\bf Screen} object like this.
\\
\begin{lstlisting}
  int fd = screen->_fd = open( "/dev/fb0", ... );
  screen->frame_buffer = (uint8_t*)mmap( 0, SCREEN_SIZE, ... , MAP_SHARED, fd, 0 );
\end{lstlisting}
The screen has a method {\bf ScreenFlush} to flush the whole internal buffer into the frame buffer:
\\
\begin{lstlisting}
  void ScreenFlush( Screen *screen ) {

    for (int i=0; i < SCREEN_SIZE; i++) {
      screen->frame_buffer[i] = screen->internal_buffer[i];
    }

  }
\end{lstlisting}
To draw to the internal buffer it supplies the {\bf ScreenDrawPixel} method:
\\
\begin{lstlisting}

  void ScreenDrawPixel(Screen *screen, int x, int y, Pixel *pixel) {

    if (x >= 0 && x < screen->width && y >= 0 && y < screen->height) {
      if (PixelNotTransparant(pixel)) {
	int coordinate = y * screen->width * 3 + x * 3;
	*((Pixel*)&screen->internal_buffer[coordinate]) = *pixel;
      }
    }
  }
\end{lstlisting}
The {\bf PixelNotTransparant} function lets us paint pictures with
transparency. If it returns {\bf false} the pixel is simply not drawn.
The function also does boundry checks to make sure that the pixel is
on the screen. This helps the graphical objects to not care about
whether they
are on the screen or not in their paint method.

\subsubsection{Canvas}
The {\bf CanvasPaint} method is found in the core of the {\bf Canvas} object.
It is reponsible for drawing each graphical object on to the
screen. This loop runs though all objects and ask them to paint themselves
 on the screen.
\\
\begin{lstlisting}
  for (int i=0; i<canvas->top; i++) {
    if (canvas->shapes[i]) {
      Shape *shape = (Shape*)canvas->shapes[i];

      (*((Shape*)shape)->paint)( canvas->shapes[i], screen );
    }
  }
\end{lstlisting}
Afterwards, the method asks the {\bf Screen} object to update the screen
buffer through the {\bf ScreenFlush} method.

\subsubsection{Shape}
The {\bf Shape} exists only to provide a structure for which other
graphical objects shall match. In this implementation the most
importaint part is the {\bf paint} function pointer, which all
graphical object must have as their second member. The first member is
a pointer to the objects parent. This is only used for the {\bf Bitmap}
type.
\\
\begin{lstlisting}
  typedef struct {
    void *parent;
    void (*paint) ( void*, Screen* );
    int x;
    int y;
    void *child;
  } Shape;
\end{lstlisting}

\subsubsection{Image}
The {\bf Image} object is only an abstract object, its responsibility
is to conseal the implemented file format. Therefore, its implementetion
is fairly simple. The {\bf paint} method delegates all its work to the
format specific object.
\\
\begin{lstlisting}
  static void paint (void *shape, Screen *screen) {

    Image *image = (Image*) shape;
    Shape *actual_image = (Shape*)image->image;
    actual_image->paint( actual_image, screen  );

  }
\end{lstlisting}

\subsubsection{Bitmap}
The {\bf Bitmap} object is by far the most complex object in the
graphics module. It is responsible for reading the bitmap fileformat
and rendering its contents in the internal buffer. The reading is
implemented in three steps.

\begin{enumerate}
\item Reading the file header \newline
  The file header is read directly into the {\bf BMPHeader} structure.
  \begin{lstlisting}
    static BMPHeader ReadBMPHeader ( int fd ) {

      BMPHeader header;
      read( fd, &header, sizeof(header));

      return header;
    }
  \end{lstlisting}

\item Reading the DIB header \newline
  We only need the first three fields of the DIB header, these are read
  into the struct.
  \begin{lstlisting}
    static DIBHeader ReadDIBHeader ( int fd ) {

      DIBHeader header;
      read( fd, &header, sizeof(header));
      return header;
    }
  \end{lstlisting}

\item Converting endian of the file header \newline
  The fileformat headers are written in little-endian, the Linux on the
  board uses big-endian. This is fixed with a bit of byte swapping.
  \begin{lstlisting}
    static void fix_endian( BMPHeader *bmp_header, DIBHeader *dib_header ) {

      char *p;

      // BMP header
      p = (char*) &bmp_header->signature;
      swap( &p[0], &p[1] );

      p = (char*) &bmp_header->size;
      swap( &p[0], &p[3] );
      swap( &p[1], &p[2] );

      p = (char*) &bmp_header->reserved1;
      swap( &p[0], &p[1] );

      p = (char*) &bmp_header->reserved1;
      swap( &p[0], &p[1] );

      p = (char*) &bmp_header->offset;
      swap( &p[0], &p[3] );
      swap( &p[1], &p[2] );

      // DIB header
      p = (char*) &dib_header->size;
      swap( &p[0], &p[3] );
      swap( &p[1], &p[2] );

      p = (char*) &dib_header->width;
      swap( &p[0], &p[3] );
      swap( &p[1], &p[2] );

      p = (char*) &dib_header->height;
      swap( &p[0], &p[3] );
      swap( &p[1], &p[2] );

    }
  \end{lstlisting}

\item Making sense of the headers
  \begin{lstlisting}
    int offset = bmp_header.offset;
    int size = bmp_header.size;
    int width = dib_header.width;
    int height = dib_header.height;
  \end{lstlisting}
\item Reading the content \newline
  The actual bitmap from the offset in the header untill the end of the
  file. All of this is read into the {\bf data} variable.
  \begin{lstlisting}
    uint8_t *data = malloc(size);
    uint8_t *ptr = (uint8_t*)data;

    for (int i=0; i < (size - offset) / BUFFER_SIZE; i++ ) {
      read( fd, buffer, BUFFER_SIZE );

      for (int i=0; i<BUFFER_SIZE; i++) {
	(*ptr++) = buffer[i];
      }
    }

    int remaining = (size - offset) % BUFFER_SIZE;
    if (remaining) {
      read( fd, buffer, remaining);

      for (int i=0; i<remaining; i++) {

	(*ptr++) = buffer[i];
      }
    }
  \end{lstlisting}

  \item Flipping the rows of the content \newline
The rows in the bitmap array is stored in the wrong order so
we have to flip them over.
    \begin{lstlisting}
      static void swapLine ( uint8_t *a, uint8_t *b, int width ) {

	for (int i=0; i<width; i++) {
	  swap(&a[i], &b[i]);
	}

      for (int i=0; i<width; i++) {
	swap(&a[i], &b[i]);
      }

    }

    static void flip ( uint8_t *data, int width, int height ) {

      for (int i=0, j=height-1 ; i < height/2; i += 1, j -= 1) {
	swapLine( &data[i*width*3], &data[j*width*3], width*3 );
      }
    \end{lstlisting}
  \item Saving the useful data\newline
Now we only need the {\bf width}, {\bf height} and {\bf data} in order
to draw the image on the screen.
    \begin{lstlisting}
	bmp->width = width;
	bmp->height = height;
	bmp->pixels = data;
    \end{lstlisting}
\end{enumerate}
The {\bf paint} method of the {\bf Bitmap} object copies the pixel
values from its internal memory. It draws them on the screen with a
vertical and horizontal shift according to its parent, {\bf Image}, x
and y values. This enables the image to be painted at an arbitrary
place on the screen.
\\
\begin{lstlisting}
  static void paint ( void *shape, Screen *screen ) {

    Bitmap *image = (Bitmap*) shape;
    Image *parent = (Image*) image->parent;

    int sx = parent->x;
    int sy = parent->y;

    int width = image->width;
    int height = image->height;

    for (int y = 0; y < height; y++) {
      for (int x = 0; x < width; x++) {
	Pixel *p = &image->pixels[y*width + x];
	ScreenDrawPixel( screen, x+sx, y+sy, p);
      }
    }
  }
\end{lstlisting}

\subsubsection{Line}
The {\bf Line} object is responsible for drawing lines on the {\bf Screen}. It can be
used for making compound objects like rectangles and triangles. The paint method uses
Simplified Bresenham's line algorithm\footnote{http://en.wikipedia.org/wiki/Bresenham's\_line\_algorithm}.

\begin{lstlisting}
  while (1) {
    ScreenDrawPixel(screen, x0, y0, &pixel);
    if (x0 == x1 && y0 == y1) {
      break;
    }

    int e2 = 2*err;
    if (e2 > -dy) {
      err = err - dy;
      x0 = x0 + sx;
    }
    if (e2 < dy) {
      err = err + dx;
      y0 = y0 + sy;
    }
  }

\end{lstlisting}

\subsubsection{Rectangle}
The {\bf Rectangle} object uses the {\bf Line} object to draw a rectangle on the {\bf Screen}.
\begin{lstlisting}
  static void paint ( void *shape, Screen *screen ) {

    Rectangle *rect = (Rectangle*)shape;

    int x = rect->x;
    int y = rect->y;
    int dx = rect->dx;
    int dy = rect->dy;

    Line l1 = LineNew(x, y, dx, 0);
    l1.paint( &l1, screen );

    Line l2 = LineNew(x, y+dy, dx, 0);
    l2.paint( &l2, screen );

    Line l3 = LineNew(x, y, 0, dy);
    l3.paint( &l3, screen );

    Line l4 = LineNew(x+dx, y, 0, dy);
    l4.paint( &l4, screen );
  }
\end{lstlisting}
