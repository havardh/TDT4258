\section{Introduction}
This is the techinical report for the third and
final exercise in the course TDT4258.
We were already fairly known with the STK1000 
microcontroller after the previous exercises, including the
GNU toolchain and C programming.\\
\\
The new concept of this exercise was the use of device 
drivers for the different devices on the development board.
So the main task was to understand how the Linux device
drivers is built and how they work in order to create our
own drivers and use the already existing drivers in a C program.\\
\\
The first part of the report explains the setup on the development
board and how we got the Linux operating system to work on it. It
also explains how we developed the code, compiled it for the AVR32 
microprocessor and ran it on the microcontroller.\\
\\
The second part roughly explains how we implemented the two drivers
for leds and buttons, respectively, and how they work with the rest
of the system.
It also explains the two frameworks that we built around the 
graphics device and the sound device, so that we could 
draw pictures of the .bmp format and play .wav sound files without
any trouble. Lastly it gives a short overview over the game and how
it works.\\
\\
The third part shows how we solved the different problems, and what
we had to go through in order to get the different devices to work 
with the game.
We use some code examples where we see fit, to explain the 
implementation in more detail.\\
\\