\section{Overview of the Solution}

\subsection{States}

\subsubsection{Modes}

The program has two main modes. These are Piano and Playback. To switch between state the \textbf{SW0} is
pressed.

\begin{figure}[h]
  \centerline{\includegraphics[width=150px]{mode.png}}
  \caption{State diagram for modes}
\end{figure}

\subsection{Datastructures}

In order to play sounds the sound samples has to be saved. A sample contains four tracks where each track
is a series of notes in a linked list.

\subsubsection{Note structure}
A musical note is represented in the program by the datatype note\_t defined in \textbf{note.h}.
\begin{figure}[h]
  \centerline{\includegraphics[width=80px]{note_t.png}}
  \caption{Datatype for notes}
\end{figure}

This type is used as a linked list to produce a track. The pitch is a number which relates to the frequency
of the note. Pitches are defined in \textbf{tone.h} (C, D, .., C2, C3, ...). The duration is how long the note lasts, durations
are also defined in \textbf{tone.h} (WHOLE, HALF, FORTH, etc.). The pogress field is a state variable used
when the tone is played in order to know when its done, its always set to 0. The cutoff is to let different tones have different
quality. The range of the cutoff is 0.0 - 1.0 where 0.5 is a very \textit{staccato}\footnote{A shortened duration of the tone}
tone and 1.0 is \textit{glissando}\footnote{When tones glides into each other}. The value 0.875 is used
for ordinary notes.

\subsubsection{Sound Tracks}
The \textbf{playback.c} file contains a array, \textit{tracks}, of constant size 4. This array has pointers to
the current point of each track. A track is a linked list of notes. The list is NULL terminated.
As there are 4 tracks, a sound sample can play four tones at a time.
\begin{figure}[h]
  \centerline{\includegraphics[width=300px]{tracks.png}}
  \caption{Illustration of tracks setup with 2 tracks}
\end{figure}

\subsubsection{Sound Samples}
The 7 different soundsamples are contained within a array of function pointers inside the interrupt.c file.
Each function initilizes the tracks array by using the \textbf{set\_track} function.

\subsubsection{Sine Table}
As the abdac interrupt routine is on a deadline, it has to conserve it's computing. Computing a \textbf{sin}
function on every interrupt is wastefull and to timeconsuming. To make this a constant operation at runtime
a sine table is computed on startup and stored in the \textbf{sine\_table} in the \textbf{samples.c} file.

\subsection{Waves}

To make sound waves we need some functions producing wave signals. The following waves are implemented
in \textbf{samples.c}.

\begin{figure}[h]
\centering
  \subfigure[Sine wave]{\includegraphics[width=150px]{sine.png}}
  \subfigure[Square wave]{\includegraphics[width=150px]{square.png}}
  \subfigure[Sawtooth wave]{\includegraphics[width=150px]{sawtooth.png}}
  \subfigure[Triangle wave]{\includegraphics[width=150px]{triangle.png}}
\end{figure}
