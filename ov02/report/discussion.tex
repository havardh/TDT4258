\section{Discussion}
This assignment was easier getting started with, because we knew the tools,  
and the programming language.
Of course handling I/O devices without an Operating System was a new 
experience. We found that having a some
knowledge of music- and wave-theory quite favorable, but the recitation 
slides and the compendium worked as
good summary and reinstatements of the theory.

\subsection{Deadline}
Deadlines are the time restriction on the interrupt routine. The ABDAC interrupts 
with a frequency of 81.920 kHz.
If it does not get a new sample in SDR before 256 cycles then we have missed a 
deadline. When we compile with
different optimizations flags on gcc (-O0, -O1, -O2, and -O3) we get slightly 
different sounds and speeds of the
audio samples. This indicates that our interrupt routine is to long and that 
we are missing some deadlines.
We found that -O1 gave us the best result.

\subsection{ABDAC interrupt routine and Sample format}
We choose to save the sound samples as linked lists. This gave us compact storage, flexible notation (tracks are not
synchronous), and little overhead at startup. But it means that the interrupt routine has a lot of work to do when it
plays a sample. If we had stored the sound samples as an array of the actual samples to be played (which could be built
on startup or even precompiled off target), the interrupt routine could be simplified to:

\begin{lstlisting}
// Predefined
#define N 100000
static int16_t samples[N] = { ... };

__int_handler *abdac_isr( void ) {
    static i = 0;
    if ( i < N ) {
      return samples[i];
      i += 1;
    } else {
      return 0; // silence
    }
}
\end{lstlisting}
This would have been an simpler design and would have enabled us to easier support sound samples from know formats
as wav and midi. This would also have helped on meating the deadlines.
